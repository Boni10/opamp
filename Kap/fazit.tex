\chapter{Fazit}

In diesem Versuch wird wieder einmal mehr die Bedeutung der Regel- und Messtechnik veranschaulicht. 
Mit ein paar wenigen Bauteilen lassen sich also Herzschläge von Menschen messen. 
Nötig dafür sind lediglich Operationsverstärker und ein paar Filter, um ein besseres Signal zu erhalten. 
Mit zwei Klemmen und über die Verschaltung der Operationsverstärker wurde der Herzschlag von uns Beiden aufgenommen.
Digital umgewandelt und über FFT gefiltert ist es uns gelungen klare Peaks des Herzausschlags zu messen.
Die Auswertung der Herzfrequenz ist, nach Vergleich mit anderen Herzfrequenzmessern, sehr exakt.
Wir haben auch Kennlinien von den Filterbauteilen Tief-, Hoch- und Bandpass aufgenommen.
Ein Vergleich im Internet zeigte hier, dass unsere Kennlinien falsch waren. Vermutung liegt bei: Beschädigter Ware.
Denn ein selbstgebauter Hochpass funktionierte hier einwandfrei, bei selbiger Messmethode.
Ebenso haben wir Kennlinien eines Operationsverstärkers mit verschiedener Verstärkung gemessen, dabei gab es bei unseren Messungen keine Probleme, die Kennlinien sind annhändernd Geraden. 
Diese Geraden geben auch exakt wieder, was der Operationsverstärker optimalerweise erfüllen soll.

Der Versuch ist weitgehend an Elektrotechnik orientiert, man verbringt die meiste Zeit damit,
die einzelnen Bauteile auf einem Schaltbrett herumzustecken. 
Es war dennoch sehr interessant und gibt grundlegendes Verständnis der Regel- und Messtechnik.\\
%Fänd ich noch schön als vergleichsbild zu unseren Herzschlägen.
%https://de.depositphotos.com/114828470/stock-illustration-heartbeat-icon-vector-illustration.html
